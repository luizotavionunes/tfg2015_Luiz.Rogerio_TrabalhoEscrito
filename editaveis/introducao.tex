\chapter[Introdução]{Introdução}
A busca por maneiras mais eficientes\footnote{Compreende-se como aumento de eficiência todos os procedimentos que contribuam para o
aumento de lucro da empresa.} de produção e gerenciamento no ambiente empresarial sempre foi um esforço constante. Mais do que uma 
questão de aumento de eficiência, no cenário atual, isso se tornou uma questão de sobrevivência no mercado. Um dos itens chave na busca 
do aumento da eficiência no meio industrial e comercial é a automatização de processos. 

O sucesso de uma instituição/empresa está diretamente relacionado a forma com que o administrador/empresário otimiza os processos nela 
envolvidos. De maneira geral tem-se a seguinte situação: quanto maior o nível de automatização, maior será o seu faturamento bem como a
suas margens de lucro. Além disso, o seu nível de profissionalização tende a se elevar. A partir da década de 70 vários setores da economia 
viram-se obrigados a seguir essa linha de pensamento, realizando a implantação em massa de técnicas de otimização em seus processos produtivos. 
De acordo com os dados da empresa de associação comercial de varejo \texttt{National Retail Federation} divulgados por \cite{beitol2015}, as 
lojas que utilizaram equipamentos e \textit{softwares} com o intuito de automatizar a captura de dados no \texttt{Ponto de Venda (PDV)}, obtiveram em 
média 16\% de aumento de vendas, com algumas lojas atingindo até 23\%.

Processos que são operados e/ou controlados exclusivamente por funcionários, estão sujeitos a vários riscos e inconvenientes como por exemplo: 
erros provenientes de falha humana, informações insuficientes e/ou inconsistentes à respeito do processo de produção, dificuldade ou mesmo a
impossibilidade de realização de um \textit{feedback} confiável, queda de produtividade e consequentemente de lucros, entre outros.

A engenharia de controle ao lado da computação tiveram papel fundamental na solução desse problema, sendo responsáveis pelo surgimento dos 
conceitos de automação industrial e informatização pela primeira vez. Nos setores primários e secundários da economia, prevalece o conceito 
de automação industrial, que consiste em sistemas automáticos de \textit{hardware} e/ou \textit{software}. Já as soluções de automatização 
por \textit{software} são amplamente utilizadas no setor terciário. Independentemente de qual setor da economia que a instituição se enquadre, 
o objetivo de tais sistemas é justamente eliminar ou pelo menos amenizar os problemas citados acima. 

No meio industrial prevalece a utilização de computadores de pequeno porte, que tem como características o seu baixo poder de processamento 
(no entanto, capacidade suficiente para a execução da tarefa desejada) e baixo custo. Tais computadores são mais conhecidos como microcontroladores,
que são compostos por uma unidade central de processamento, memória, e pinos de \texttt{I/O}. Muitos modelos de microcontroladores vão além 
dessa estrutura básica, apresentando uma série de periféricos integrados. Entretanto, o processo de implantação de alguma solução de automação 
ou informatização, requer um estudo detalhado de como projetar, avaliar e adquirir os componentes necessários a implementação do sistema automatizado 
(sistemas automatizados podem ser constituídos tanto por soluções de \textit{hardware}, \textit{software} ou de ambas).

  	A utilização de microcontroladores em conjunto com algum \textit{software} que tenha como objetivo a coleta, análise e armazenagem de 
  	dados faz surgir o conceito de sistema supervisório. Em outras palavras, tal sistema consiste em uma ferramenta capaz de adquirir e armazenar
  	informações, além de poder controlar uma grande variedade de processos. 

O presente trabalho irá abordar a implementação de um sistema supervisório de tempo real, destinado ao controle do fluxo de clientes em uma 
rede de motéis. A problemática do trabalho foi construída a partir de uma situação indesejada enfrentada por um empresário proprietário de uma
rede de motéis na qual, o controle de entrada e saída de clientes é totalmente realizado de forma manual. O empresário passava por vários problemas 
oriundos dessa abordagem como por exemplo: falta de controle sob a entrada e saída clientes, o que provocava a geração de um fluxo de caixa 
inconsistente, além de extravios de dinheiro por parte dos funcionários responsáveis pelo controle do processo. Consultando várias soluções 
oferecidas pelo mercado, o empresário sentiu-se desmotivado em função do elevado custo relacionado a aquisição e implantação do sistema. 
Diante de tal situação, é formulada a seguinte problemática para o trabalho em questão: 
“Como implementar um sistema supervisório de tempo real de baixo custo para realizar o controle do fluxo de clientes na rede de móteis? ” 

	O sistema a ser desenvolvido terá a sua parte de aquisição de dados constituída por um computador de baixa capacidade e sensores de 
	proximidade que serão responsáveis pela detecção do fluxo de entrada e saída de clientes. A parte de armazenamento de informações será
	implementa por meio de um banco de dados online. A manipulação e análise dos dados será realizada por um \textit{software} a ser 
	desenvolvido que permitirá a visualização e interpretação dos dados adquiridos em tempo real.
